\section{The Grammar}
\label{lbl:grammar}

In this section, we mainly focus on the syntax of our language, it is divided into different parts and basically described in  \emph{Extended Backus-Naur form (EBNF)}.

Let's introduce the overview of this language first.

\begin{bnf}
    \ntsym{program} &::= & [\ntsym{statement}]^* \\
    \ntsym{statement} &::=& \ntsym{importStmt} | \ntsym{typedefStmt} | \ntsym{functionStmt} \\
    &|& \ntsym{channelStmt} | \ntsym{componentStmt} \\
    &|& \ntsym{connectorStmt} \\
\end{bnf}

\subsection{Type System}

The basic idea behind this type system is that we try to satisfy both researchers and programmers.

\begin{bnf}
    \ntsym{primitiveType} &::= & \tsym{int}[\ntsym{term}\tsym{..}\ntsym{term}] \\
    & | & \tsym{double} | \tsym{char} | \tsym{bool} | \tsym{NULL} \\
    & | & \tsym{enum} {\{} \ntsym{identifier}^+\tsym{\}} \\ 
    & | & \ntsym{identifier} \\
    \ntsym{extendedType} &::= & \ntsym{primitiveType} \\
    & | & \ntsym{type} \tsym{$|$} \ntsym{type} \\
    & | & \tsym{array} \ntsym{extendedType} \tsym{[} \ntsym{term} \tsym{]}\\
    & | & \tsym{map} \tsym{[} \ntsym{extendedType}\tsym{]} \ntsym{type}\\
    & | & \tsym{struct} \tsym{\{} (\ntsym{identifier} \tsym{:} \ntsym{type})^+ \tsym{\}}\\
\end{bnf}

% TODO is raw type a common name ? needs some googling

\begin{definition}[Finite Types]
    A type in \lang{} is \emph{finite} iff. it has a finite value domain.
\end{definition}


\noindent\emph{Primitive Type.} \lang{} provides built-in support for common primitive types, they are,
\begin{itemize}
    \item \emph{Int}. Similar to  
    \item \emph{Double.}
    \item \emph{Bool.} Boolean variables.
    \item \emph{Enum.} An \emph{enumeration} is a finite collection of unique identifiers.
    \item \emph{NULL.} For simplicity, in \lang{} we don't have pointers references, however, sometimes an empty value is strongly required to denote uninitialized value. Consequently, we define a single-value type NULL, of which the only possible value is \emph{null}.
\end{itemize}

\noindent\emph{Extended Type.} Extended type offers an approach to contruct complex data types with simpler ones. Four extending patterns are introduced as follows,
\begin{itemize}
    \item \emph{Union}. The \emph{union} operator `$|$' is designed to combine two \emph{disjoint} types as a more complicated one. This is similar to the union type in C language but much easier to use.
    \item \emph{Array} and \emph{Slice}. An \emph{array} $T$[n] is a finite ordered collection containing exactly $n$ elements of type $T$. Moreover, a \emph{slice} is an array of which the capacity is not specified.
    \item \emph{Map}. A \emph{map }[$T_{key}$] $T_{val}$ is a dictionary that maps a key of type $T_{key}$ to a value of type $T_{val}$.
    \item \emph{Struct}. A \emph{struct }\{$field_1:T_1,\cdots,field_n:T_n$\} contain $n$ fields, each has a particular type $T_i$ and a unique identifier $id_i$.
\end{itemize}

Type systems often differ greatly between formal models and real-world programming languages. For example, PRISM\cite{KwiatkowskaCav2011}, ..

% TODO: in popular standards, a system may contains different safety/security level. 

In \lang's type system, most of basic types are finite except for the unlimited integers and doubles. Besides, extended types are also finite if they are based on finite ones.

% typing rules

\begin{definition}[Subtyping]
    Type $T_1$ is a subtype of $T_2$, denoted by $T_1\subtype
 T_2$, iff. a value of type $T_1$ can be converted to a value of type $T_2$ losslessly.
\end{definition}

\begin{mathpar}
    \inferrule* [right=S-FiniteInt] {l_1,u_1,l_2,u_2 \in \mathbb{Z}, l_1\leq u_1\land l_2\geq u_2}{\mbox{int $l_1$ .. $u_1$} \subtype \mbox{int $l_2$ .. $u_2$}} \\
    \inferrule* [right=S-Enum] {n \in \mathbb{Z}}{\mbox{enun \{ $item_0,\cdots,item_{n-1}$\}} \subtype \mbox{int 0 .. (n - 1)}} \\
    \inferrule* [right=S-Bool] {.}{\mbox{bool} \subtype \mbox{int 0 .. 1}} \\
    \inferrule* [right=S-Int] {l,u \in \mathbb{Z}}{\mbox{int $l$ .. $u$} \subtype \mbox{int}}\\
    \inferrule* [right=S-Union] {T_1\subtype T_3, T_2\subtype T_4}{T_1 | T_2 \subtype T_3 | T_4 } \\
\end{mathpar}

\begin{definition}[Abstract and Concrete Types]
    The types which have default values specified are called concrete types. And on contrary, those without default values are termed abstract types.
\end{definition}
\begin{bnf}
    \ntsym{concreteType} & ::= & \ntsym{extendedType} \tsym{init} \ntsym{term}
\end{bnf}

In most programming languages, a non-reference type has a built-in default value. However, such values are not specified by programmers, and sometimes may lead to unexpected behavior.

\subsection{Terms}

\begin{bnf}
    \ntsym{term} &::=& \ntsym{termPrimitive} \\
    &|& \ntsym{term} \tsym{+} \ntsym{term} | \ntsym{term} \tsym{-} \ntsym{term} \\
    &|& \ntsym{term} \tsym{*} \ntsym{term} | \ntsym{term} \tsym{/} \ntsym{term} \\
    &|& \tsym{!} \ntsym{term} \\
    &|& \ntsym{identifier} \tsym{(} \ntsym{term}^* \tsym{)} \\
    &|& \ntsym{term} \tsym{[} \ntsym{term} \tsym{]} \\
    &|& \ntsym{term} \tsym{.} \ntsym{identifier} \\
\end{bnf}

\begin{mathpar}
    \inferrule* [right=T-Call]{t_1:T_1,\cdots,t_n,T_n,f:T_1\rightarrow \cdots T_n\rightarrow T}{f(t_1,\cdots,t_n):T} \\
    \inferrule* [right=T-Array]{t_{arr}:T\:[n],0\leq i<n}{t_{arr}[i]:T} \\
    \inferrule* [right=T-Slice]{t_{slice}:T\:[\:],0\leq i< len(t_{slice})}{t_{slice}[i]:T} \\
    \inferrule* [right=T-Map]{t_{map}:\mbox{map }[T_1]\:T_2,t_{key}:T_1}{t_{map}[t_{key}]:T_2}\\
    \inferrule* [right=T-struct]{t_{struct}:\mbox{struct }\{field_1:T_1,\cdots,field_n:T_n\},1\leq i\leq n}{t_{struct}.field_{i}:T_i}
\end{mathpar}

Rules of operators (e.g. $+,-,*$) are not presented here. That's mainly due their abstract nature. In \lang{}, native semantics of numeric operators are absent on syntax level. In other words, we have to manually override these operators as functions, for example.

\begin{example}[Operator Overriding]
    \begin{lstlisting}
native function operatorAdd (a:int,b:int):int;
native function <l1:int,l2:int,u1:int,u2:int> operatorAdd(a:int l1 .. u1, b:int l2 .. u2) : int (l1 + l2) .. (u1 + u2);
    \end{lstlisting}
\end{example}

% TODO: note that in the rules there some logic formulas. in practice, these formulas needs to be derived from guard conditions or else

\subsection{Channels and Components}

Channels are the atomic functional units in our language. A channel compries several parameters; a set of ports, either \emph{input} or \emph{output}; some private variables; and a series of \emph{ordered} transitions.

Essentially, behavior of a channel is encoded as a set of \emph{guarded transitions}. 

Before introducing the formal grammar of channels, we present a simple example here.

% \begin{example}[Asynchronous Buffer Channel].

% \end{example}

% \begin{lstlisting}
% channel <type T, int n> FIFO1 (in A T, out B T)
%     variables
%         buf (T | EMPTY) init EMPTY;
%     end variables

%     transitions
%         buf != EMPTY ->
%             A.reqRead, B.reqWrite := false, true;
%         buf == EMPTY ->
%             A.reqRead, B.reqWrite := true, false;

%         A.reqWrite && buf == EMPTY ->
%         begin
%             perform A;
%             buf := A;
%         end

%         B.reqRead && buf != EMPTY ->
%         begin
%             B := buf;
%             buf := EMPTY;
%             perform B;
%         end
%     end transitions
% end channel
% \end{lstlisting}

\begin{bnf}
    \ntsym{channel} & ::= & \tsym{channel}[\ntsym{template}]\ntsym{identifier} \stsym{(} \ntsym{port}^* \stsym{)}\\
    & & \tsym{\{} \ntsym{variables} \ntsym{transitions} \tsym{\}} \\
    \ntsym{port} & ::= & \ntsym{identifier} \tsym{:} (\tsym{in}|\tsym{out}) \ntsym{type} \\
\end{bnf}

Templates make it able to create a family of similar channels with a single definition. That is, we are able to declare a set of parameters in the channel's definition. And when creating channel instances, you have to specify concrete values to these parameters. A parameter here can be either a type identifier (decorated with prefix \textbf{type}), or a normal variable. 

\begin{bnf}
    \ntsym{template} & ::= & \tsym{$\langle$} \ntsym{param}^+ \tsym{$\rangle$} \\
    \ntsym{param} & ::= & (\tsym{type} | \ntsym{type}) \ntsym{identifier}
\end{bnf}

\noindent \emph{Variables.} Local variables can be declared in the \emph{variables} segement, and referenced in the \emph{trnasitions} segement. 

\begin{bnf}
    \ntsym{variables} & ::= & \tsym{variables} \tsym{\{} (\ntsym{identifier} \ntsym{type} [\tsym{init} \ntsym{term}])^* \tsym{\}}
\end{bnf}

\noindent \emph{Transitions.} A transition is a guarded command that is executed when 
\begin{bnf}
    \ntsym{transitions} & ::= & \tsym{transitions} \\
    & & (\ntsym{transition}|\ntsym{group})^* \\
    & & \tsym{end transitions} \\
    \ntsym{transition} & ::= & \ntsym{term} \rightarrow \\
    & & (\ntsym{statement} | \tsym{begin} \ntsym{statement}^+ \tsym{end})\\
    \ntsym{statement} & ::= & \ntsym{identifier}^+\tsym{:=}\ntsym{term}^+\\
    & | & \tsym{perform} \ntsym{identifier}^+ \\
    \ntsym{group} & ::= & \tsym{group} \ntsym{transition}^+ \tsym{end group}
\end{bnf}

\subsection{Connectors}

\begin{bnf}
    \ntsym{connector} & ::= & \tsym{connector}[\ntsym{params}]\ntsym{identifier} \stsym{(} \ntsym{ports} \stsym{)} \\
    & & [\ntsym{internal}] (\ntsym{connection})^+ \tsym{end connector} \\
    \ntsym{interal} & ::= & \tsym{internal} (\ntsym{identifier})^+ \\
    \ntsym{connection} & ::= & \ntsym{identifier} \tsym{(} \ntsym{identifier}^+ \tsym{)}
\end{bnf}
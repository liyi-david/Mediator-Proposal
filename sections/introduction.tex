\section{Introduction}
\label{sec:introduction}

Component-based software engineering has been prospering for decades. Through proper encapsulations and clearly declared interfaces, \emph{component}s can be reused by different applications without knowledge of their implementation details.

Currently, there are various tool supports on component-based modeling. NI LabVIEW \cite{labview}, MATLAB Simulink \cite{hahn2016essentialsimulink} and Ptolomy \cite{KimPtolomy2017} provide powerful formalism and a large number of built-in component libraries to support commonly-used platforms. However, due to the complexity of models, such tools mainly focus on synthesis and simulation, instead of formal verification.
There are also a set of formal tools that prefer simple but verifiable model, e.g. Esterel SCADE \cite{AbdullaISoLA2006} and rCOS \cite{LiuFsenRcos2010}. SCADE, based on a synchronous data flow language LUSTRE, is equipped with a powerful tool-chain and widely used in development of embedded systems. rCOS, on the other hand, is a refinement calculus on object-oriented designs.

Existing work \cite{ZouSimulinkHcsp2013} has shown that, formal verification based on existing industrial tools is hard to realize due to the complexity and non-open architecture of these tools. Unfortunately, unfamiliarity of formal specifications is still the main obstacle hampering programmers from using formal tools. For example, even in the most famous formal modeling tools with perfect graphical user interfaces (like PRISM \cite{KwiatkowskaCavPrism2011} and UPPAAL \cite{AmnellMovepUppaal2001}), sufficient knowledge about automata theory is necessary to properly encode the models.

The channel-based coordination language Reo \cite{ArbabMscsReo2004} provides a solution where advantages of both formal languages and graphical representations can be integrated in a natural way. As an exogenous coordination language, Reo doesn't care about the implementation details of components. Instead, it takes \emph{connector}s as the first-class citizens. Connectors are organized and encapsulated through a compositional approach to capture complex interaction and communication behavior among components.

In this paper we introduce a new modeling language \emlang{}. \lang{} is a hierarchical modeling language that provides proper formalism for both high-level \emph{system} layouts and low-level \emph{automata}-based behavior units. A rich-featured type system describes complex data structures and powerful automata in a formal way. Both components and connectors can be declared through automata to compose a system. Moreover, automata and systems are encapsulated with \emph{a set of input or output ports} (which we call an \emph{interface}) and \emph{a set of template parameters} so that they can be easily reused in multiple applications. 

The paper is structured as follows. In Section \ref{sec:syntax}, we briefly present the syntax of \lang{} and formalizations of the language entities. Then in Section \ref{sec:semantics}. we introduce the formal semantics of \lang{}. Section \ref{sec:casestudy} provides a case study where a commonly used coordination algorithm \emph{leader election} is modeled in \lang{}. Section \ref{sec:conclusion} concludes the paper and comes up with some future work we are going to work on.
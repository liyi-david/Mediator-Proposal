\section{Semantics}
\label{sec:semantics}

In this section, we introduce the formal semantics of \lang{} through the following steps. First we use the concept \emph{configuration} to describe the state of an automaton. Next we show what the canonical forms of the transitions and automata are, and how to make them canonical. Finally, we define the formal semantics of automata as \emph{labelled transition systems (LTS)}.

Instead of formalizing systems as LTS directly, we propose an algorithm that flattens the hierarchical structure of a system and generates a corresponding automaton.

\subsection{Configurations}
\label{subsec:config}
States of a \lang{} automaton depend on the values of its \emph{local variables} and \emph{port variables}. First we introduce the definition of \emph{evaluation} on a set of variables. 

\begin{definition}[Evaluation]
An evaluation of a set of variables $V$ is defined as a function $v:V\rightarrow \mathbb{D}$ that satisfies $\forall x\in V,v(x)\in Dom(type(x))$. We denote the set of all possible evaluations of $Vars$ by $EV(Vars)$.
\end{definition}

Basically, an evaluation is a function that maps variables to one of its valid values, where we use $\mathbb{D}$ to denote the set of all values of all supported types. Now we can introduce \emph{configuration} that snapshots an automaton.

\begin{definition}[Configuration] A configuration of an automaton $A=\langle Ports,$ $Vars,Trans_G\rangle$ is defined as a tuple $(v_{loc},v_{adj})$ where $v_{loc}\in EV(Vars)$ is an evaluation on local variables, and $v_{adj}\in EV(Adj(A))$ is an evaluation on port variables. We use $Conf(A)$ to denote the set of all configurations of $A$.
\end{definition}

Now we can mathematically describe the language elements in an automaton:
\begin{itemize}
    \item \emph{Guard}s of an automaton $A$ are represented by boolean functions on its configurations $g:Conf(A)\rightarrow Bool$.
    \item \emph{Assignment Statement}s of $A$ are represented by functions that map configurations to their updated ones $s_a:Conf(A)\rightarrow Conf(A)$.
\end{itemize}

\subsection{Canonical Form of Transitions and Automata}
\label{subsec:canonical}

Different statement combinations may have the same behavior. For example, \texttt{a := b; c := d} and \texttt{a, c := b, d}. Such irregular forms may lead to an extremely complicated and non-intuitive process when joining multiple automata. 
To simplify this process, we introduce the \emph{canonical} form of transitions and automata as follows.

\begin{definition}[Canonical Transitions]
A transition $t=g\rightarrow[s_1,\cdots,s_n]$ is canonical iff. its statements are a non-empty interleaving sequence of assignments and synchronizing statements which starts and ends with assignments.
\end{definition}

Suppose $g\rightarrow[s_1,\cdots,s_n]$ is a transition of automaton $A$, it can be made canonical through the following steps.
\begin{description}
    \item[S1] If we find a continuous subsequence $s_i,\cdots,s_j$ (where $s_k$ is an assignment statement for all $k= i,i+1,\cdots,j$, and $j>i$), we merge them as a single one. Since the assignment statements are formalized as functions $Conf(A)\rightarrow Conf(A)$, the subsequence $s_i,\cdots, s_j$ can be replaced by $s'=s_j\circ\cdots \circ s_i$\footnote{The symbol $\circ$ denotes the composition operator on functions.}.
    \item[S2] Keep on going with \emph{S1} until there is no further subsequence to merge.
    \item[S3] Use identical assignments $id_{Conf(A)}$ to fill the gap between any adjacent synchronizing statements. Similarly, if the statements' list starts or ends with a synchronizing statement, we should also use $id_{Conf(A)}$ to decorate its head or tail.
\end{description}

It's clear that once we found such a continuous subsequence, the merging operation will reduce the number of statements. Otherwise it stops. It's clear that $S$ is a finite set, and the algorithm always terminates within certain time.

\begin{definition}[Canonical Automata]
    $A=\langle Ports,$ $Vars,Trans_G\rangle$ is a cano-nical automaton iff. a) $Trans_G$ includes only one transition group and b) all transitions in this group are canonical.
\end{definition}

Now we show for an arbitrary automaton $A=\langle Ports,$ $Vars,Trans_G\rangle$, how $Trans_G$ is reformed to make $A$ canonical.
Suppose $Trans_G$ is a sequence of transition groups $t_{G_i}$, where the length of $t_{G_i}$ is denoted by $l_i$,
\[
    [t_{G_1}=\{g_{11}\rightarrow S_{11},\cdots, g_{1l_1}\rightarrow S_{1l_1}\},\cdots,t_{G_n}=\{g_{n1}\rightarrow S_{n1},\cdots,g_{nl_n}\rightarrow S_{nl_n}\}]
\]

Informally speaking, once a transition in $t_{G_i}$ is activated, all the other transitions in $t_{G_j}(j>i)$ are strictly prohibited from being fired. We use $activated(t_G)$ to denote the condition where at least one transition in $t_G$ is enabled, formalized as
\[
    activated(t_G=\{g_1\rightarrow S_1,\cdots, g_n\rightarrow S_n\}) = g_1\lor\cdots\lor g_n
\]

To simplify the equations, we use $activated(t_{G_1},\cdots,t_{G_{n-1}})$ to indicate that at least one group in $t_{G_1},\cdots,t_{G_{n-1}}$ is activated. It's equivalent form is:
\[
    activated(t_{G_1})\lor\cdots\lor activated(t_{G_{n-1}})
\]
Then we can generate the new group of transitions with no dependency on priority as followings.
\begin{eqnarray*}
    Trans_G' & = [ & g_{11}\rightarrow S_{11}, \cdots ,g_{1l_1}\rightarrow S_{1l_1}, \\
    & & g_{21}\land \lnot activated(t_{G_1})\rightarrow S_{21}, \cdots, g_{2l_2} \land \lnot activated(t_{G_1})\rightarrow S_{2l_2}, \cdots \\
    & & g_{n1}\land \lnot activated(t_{G_1},\cdots,t_{G_{n-1}})\rightarrow S_{n1}, \cdots, \\
    & & g_{nl_n} \land \lnot activated(t_{G_1},\cdots, t_{G_{n-1}})\rightarrow S_{nl_n}]
\end{eqnarray*}


\subsection{From System to Automaton}
\label{subsec:composition}

\lang{} provides an approach to construct hierarchical system models from automata. In this section, we present an algorithm that flattens such a hierarchical system into a typical automaton.

For a system $S=\langle Ports, Entities, Internals, Links\rangle$, Algorithm \ref{alg:compose} flattens it into an automaton $A_S=\langle Ports,Vars',Trans_G'\rangle$, where we assume that all the entities are canonical automata (\liyi{they will be flattened recursively first if they are systems}). The whole process is mainly divided into 2 steps:
\liyi{
\begin{enumerate}
    \item Rebuild the structure of the flattened automaton, i.e. to integrate local variables and resolve the internal nodes.
    \item Put the transitions together, including both internal transitions and external transitions according to the connections.
\end{enumerate}
}

\liyi{First of all, we refactor all the variables in all entities (in $Entities$) to avoid name conflicts, and add them to $Vars'$. Besides, all internal nodes are resolved in the target automaton, and be represented as
$
    \{i\_field|i\in Internals, field\in\{\texttt{reqRead, reqWrite, value}\}\}\subseteq Vars'
$
}

\liyi{Now with all local variables needed well prepared, we then start with merging the transitions, both \emph{internal} ones and \emph{external} ones.}

\begin{itemize}
    \item Internal transitions are easy to handle. Since they do not synchronize with other transitions, we directly put all the internal transitions in all entities into the flattened automaton, also as internal transitions.
    \item External transitions, on the other hand, have to synchronize with its corresponding external transitions in other entities. For example, when an automaton reads from an input port $P_1$, there must be another automaton which is writing to its output port $P_2$, where $P_1$ and $P_2$ are welded in the system. \liyi{An example is presented as follows.}
\end{itemize}

\begin{example}[Synchronizing External Transitions]
    \liyi{Let's consider two queues that cooperates on a shared internal node: \texttt{Queue(A,B)} and \texttt{Queue(B,C)}. The dequeue operation of \texttt{Queue(A,B)} and enqueue operation of \texttt{Queue(B,C)} should obviously be synchronized and scheduled. During the synchronization, the basic principle is to make sure that synchronizing statements on the same ports should be aligned strictly.}
    \begin{center}
    \begin{minipage}[t]{0.32\linewidth}
        \small{\texttt{Dequeue Operation:}}
        \begin{lstlisting}[basicstyle=\scriptsize\ttfamily,numbers=none,xleftmargin=0pt]
(B.reqRead && B.reqWrite)-> {
    B.value := buf[ptail];
    ptail := next(ptail);
    @sync B;@ <---- sync with --

    
    
}
        \end{lstlisting}
    \end{minipage}
    \begin{minipage}[t]{0.32\linewidth}
        \small{\texttt{Enqueue Operation:}}
        \begin{lstlisting}[basicstyle=\scriptsize\ttfamily,numbers=none,xleftmargin=0pt]
(B.reqRead && B.reqWrite)-> {


--> @sync B;@ <--- and goes to -

    buf[phead] := B.value;
    phead := next(phead);
}
        \end{lstlisting}
    \end{minipage}
    \begin{minipage}[t]{0.32\linewidth}
        \small{\texttt{After Scheduling:}}
        \begin{lstlisting}[basicstyle=\scriptsize\ttfamily,numbers=none,xleftmargin=0pt]
(B_reqRead && B_reqWrite)-> {
    B_value:=buf1[ptail1];
    ptail1:=next(ptail1);
--> @B_reqRead,B_reqWrite:=false,false;@
    buf2[phead2]:=B_value;
    phead2:=next(phead2);
}
        \end{lstlisting}
    \end{minipage}
    \end{center}
    
    \liyi{During the synchronization, we refactor the local variables \texttt{ptail, phead} and \texttt{buf}, and transfer internal node $B$ to a set of local variables. Synchronizing statement \texttt{sync B} is aligned between two transitions and in turn lead to the final result, where scheduled synchronizing statements are replaced by its local behavior -- to reset its corresponding port variables .}
\end{example}


\liyi{Now we formally present the flatting algorithms for systems.} In the following we use $\mathcal{P}(A)$ to denote the power set of $A$.

\begin{algorithm}[t]
    \caption{\texttt{Flat} a System into an Automaton}
    \label{alg:compose}
    \algsetup{linenosize=\small}
    \small
    \begin{algorithmic}[1]
        \REQUIRE A system $S=\langle Ports, Entities, Internals, Links\rangle$
        \ENSURE An automaton $A$
        \STATE $A \leftarrow $ an empty automaton
        \STATE $A.Ports\leftarrow S.Ports$
        \STATE $Automata\leftarrow$ all the flattened automata of $S.Entities$
        \STATE rename \emph{local variables} in $Automata=\{A_1,\cdots,A_n\}$ to avoid duplicated names
        \FOR {$l=\langle p_1,p_2\rangle\in S.Links$}
            \IF {$p_1\in S.Ports$}
                \STATE replace all occurrance of $p_2$ with $p_1$
            \ELSE
                \STATE replace all occurrance of $p_1$ with $p_2$
            \ENDIF
        \ENDFOR
        \STATE $ext\_trans\leftarrow \{\}$
        \FOR{$i\leftarrow 1,2,\cdots,n$}
            \STATE add $A_i.Vars$ to $A.Vars$
            \FOR{$internal\in A_i.Ports$}
                \STATE add $\{ internal\texttt{.reqRead}, internal\texttt{.reqWrite}, internal\texttt{.value}\}$ to $A.Vars$
            \ENDFOR
            \STATE add \emph{all internal transitions} in $A_i.Trans_G$ to $A.Trans_G$
            \STATE add \emph{all external transitions} in $A_i.Trans_G$ to $ext\_trans$
        \ENDFOR
        \FOR{$set\_trans\in \mathcal{P}(ext\_trans)$}
            \STATE add $\mbox{\texttt{Schedule}}(S, set\_trans)$ to $A.Trans_G$ \textbf{if} it is not \emph{null}
        \ENDFOR
    \end{algorithmic}
\end{algorithm}

In \lang{} \emph{systems}, only port variables are shared between automata. During synchronization, the most important principle is to make sure assignments to port variables are performed before the port variables are referenced. Basically, this is a topological sorting problem on dependency graphs. A detailed algorithm is described in Algorithm \ref{alg:synchronize}. In this algorithm, we use
\begin{itemize}
    \item $\bot$ and $\top$ to denote starting and ending of a transition's execution, \item $synchronizable$($t_1,\cdots,t_n$) to denote that the transitions are synchronizable, i.e. they come from different automaton and for each port being synchronized, there are exactly 2 transitions in $t_1,\cdots,t_n$ that synchronize it, and
    \item $reset\_stmt(p)$ to denote the corresponding statement that resets a port's status \texttt{p.reqRead, p.reqWrite := false, false}.
\end{itemize}

\begin{algorithm}[t]
    \caption{\texttt{Schedule} a Set of External Transitions}
    \label{alg:synchronize}
    \algsetup{linenosize=\small}
    \small
    \begin{algorithmic}[1]
        \REQUIRE A System $S$, a set of external canonical transitions $t_1,t_2,\cdots,t_n$
        \ENSURE A synchronized transition $t$
        \STATE \textbf{if} not $synchronizable(t_1,\cdots,t_n)$ \textbf{then} \textbf{return} $t\leftarrow null$

        \STATE \emph{t.g, t.S, G} $\leftarrow \bigwedge_i t_i.g,\:[\:]$, an empty graph $\langle V,E\rangle$
        \FOR{$i\leftarrow 1,\cdots,n$}
            \STATE add $\bot_i, \top_i$ to \emph{G.V}
            \STATE \emph{syncs}, \emph{ext\_syncs} $\leftarrow\{\bot_i\},\{\}$
            \FOR {$j\leftarrow 1, 3, \cdots, len(t_i.S)$}
                \STATE add $t_i.S_{j}$ to $G.V$
                \STATE \textbf{if} $ext\_syncs\neq\{\}$ \textbf{then} add `\texttt{sync} $ext\_syncs\mbox{'}\rightarrow t_i.S_j$ to G.E 
                \FOR {$p\in syncs$}
                    \STATE add edge $reset\_stmt(p)\rightarrow t_i.S_j$ to $G.E$
                \ENDFOR
                \STATE $syncs\leftarrow$ \{ all the synchronized ports in $t_i.S_{j+1}$ \} $\backslash S.Ports$ \STATE $ext\_syncs\leftarrow$ \{ all the synchronized ports in $t_i.S_{j+1}$ $\}\cap S.Ports$
                \IF {$j < len(t_i.S)$}
                    \FOR {$p\in $ $syncs$}
                        \STATE add $reset\_stmt(p)$ to $G.V$ if is is not included yet
                        \STATE add edge $t_i.S_j \rightarrow reset\_stmt(p)$ to \emph{G.E}
                    \ENDFOR
                    \IF {$ext\_syncs\neq \{\}$}
                        \STATE add `\texttt{sync} $ext\_syncs$' to $G.V$
                        \STATE add edge $t_i.S_j\rightarrow$ `\texttt{sync} $ext\_syncs$' to $G.E$
                    \ENDIF
                \ELSE
                    \STATE add edge $t_i.S_j\rightarrow \top_i$ to $G.E$
                \ENDIF
            \ENDFOR
        \ENDFOR
        \STATE \textbf{if} $G$ comprises a ring \textbf{then} $t\leftarrow null$
        \STATE \textbf{else} $t.S\leftarrow[$ select all the statements in $G.E$ using topological sort $]$
    \end{algorithmic}
\end{algorithm}

Algorithm \ref{alg:synchronize} may not always produce a valid synchronized transition. When the dependency graph has a \emph{ring}, the algorithm fails due to \emph{circular dependencies}.
For example, transition \texttt{g$_1$->\{sync A;sync B;\}} and transition \texttt{g$_2$->\{sync B;sync A;\}} cannot be synchronized where both $A,B$ need to be triggered first.

Topological sorting, as we all know, may generate different schedules for the same dependency graph. The following theorem shows that all the existing schedules are equivalent as transition statements.

\begin{theorem}[Equivalence between Schedules\footnote{\liyi{Proof of the theorem can be found in the appendix.}}
] If two sequences of assignment statements $S_1, S_2$ are generated from the same set of external transitions, they have exactly the same behavior (i.e. $S_1$ and $S_2$ will lead to the same result when they are executed under the same configuration).
\end{theorem}

\subsection{Automaton as Labelled Transition System}
\label{subsec:lts}

With all the language elements properly formalized, now we introduce the formal semantics of \emph{automata} based on \emph{labelled transition system}.

\begin{definition}[Labelled Transition System]
    A labelled transition system is a tuple $(S,\Sigma,\rightarrow,s_0)$ where $S$ is a set of states with initial state $s_0\in S$, $\Sigma$ is a set of actions, and $\rightarrow\subseteq S\times\Sigma\times S$ is a set of transitions. For simplicity, we use $s\xrightarrow{a} s'$ to denote $(s,a,s')\in\rightarrow$.
\end{definition}

Suppose $A=\langle Ports, Vars, Trans_G\rangle$ is an automaton, its semantics can be captured by a LTS $\langle S_A, \Sigma_A,\rightarrow_A,s_0\rangle$ where
\begin{itemize}
    \item $S_A=Conf(A)$ is the set of all configurations of $A$.
    \item $s_0\in S_A$ is the initial configuration where all variables (except for \texttt{reqRead}s and \texttt{reqWrite}s) are initialized with their default value, and all \texttt{reqRead}s and \texttt{reqWrite}s are initialized as \texttt{false}.
    \item $\Sigma_A=\{i\}\cup \mathcal{P}(Ports)$ is the set of all actions, \liyi{where $i$ denotes the internal action (i.e. no synchronization is performed)}.
    \item $\rightarrow_A\subseteq S_A\times \Sigma_A\times S_A$ is a set of transitions obtained by the following rules.
\end{itemize}

\begin{mathpar}
    \inferrule* [right=R-InputStatus] {p\in P_{in}}{(v_{loc}, v_{adj})\xrightarrow{i}_A(v_{loc},v_{adj}[p.reqWrite\mapsto \lnot p.reqWrite])} \\
    \inferrule* [right=R-InputValue] {p\in P_{in}, val\in Dom(Type(p.value))}{(v_{loc}, v_{adj})\xrightarrow{i}_A(v_{loc},v_{adj}[p.value\mapsto val])} \\
    \inferrule* [right=R-OutputStatus] {p\in P_{out}}{(v_{loc}, v_{adj})\xrightarrow{i}_A(v_{loc},v_{adj}[p.reqRead\mapsto \lnot p.reqRead])} \\
    \inferrule* [right=R-Internal] {\{g\rightarrow \{s\}\}\in Trans_G\mbox{ is internal}}{(v_{loc}, v_{adj})\xrightarrow{i}_A s(v_{loc},v_{adj})} \\
    \inferrule* [right=R-External] {g\rightarrow S\in Trans_G\mbox{ is external, } [s_1,\cdots,s_n]\mbox{ are assignments in $S$} \\ \vspace{1.5em} p_1,\cdots,p_m \mbox{ are the synchronized ports}}{(v_{loc}, v_{adj})\xrightarrow{\{p_1,\cdots,p_m\}}_A s_n\circ\cdots\circ s_1(v_{loc},v_{adj})} \\
\end{mathpar}

The first three rules describe the potential change of environment, i.e. the port variables. R-InputStatus and R-OutputStatus show that the reading status of an output port and writing status of an input port may be changed by the environment randomly. And R-InputValue shows that the value of an input port may also be updated by the environment.

The rule R-Internal specifies the internal transitions in $Trans_G$. As illustrated previously, an internal transition contains no synchronizing statement. So its canonical form comprises only one assignment $s$. Firing such a transition will simply apply $s$ to the current configuration.

Meanwhile, the rule R-External specifies the external transitions, where the automaton interact with its environment. Fortunately, since all the environment changes are captured by the first three rules, we can simply regard the environment as another set of local variables. Consequently, the only difference between an internal transition and an external transition is that the later one may contain multiple assignments.

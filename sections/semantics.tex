\section{Semantics}
\label{sec:semantics}

In the section, we introduce the formal semantics of \lang{} by four steps.
\begin{enumerate}
    \item Use \emph{configurations} to formally describe the state of a automaton.
    \item Convert an automaton to its canonical form, wiping out the dependency on language features \emph{alternative, urgent and ordered}.
    \item Provide an algorithm that flats a complex system to an automaton.
    \item Specify a transition-system-based semantics to \lang{} automata.
\end{enumerate}

% TODO: a figure is required here

\subsection{Configurations of Automata}
\label{subsec:config}
% TODO: formalization of adjoint variables
Configurations are used to represent the state of an automaton. Since we don't have locations here, it only depends on the values of its locally accessible variables, which includes both \emph{adjoint variables} and \emph{local variables}.

\begin{definition}[Valuation]
A valuation of a set of variables $V$ is defined as a function $v$ that satisfies $\forall x\in V,v(x)\in Dom(type(x))$. We denote the set of all possible valuations of $Vars$ by $Val(Vars)$.
% TODO: how to say `without confusion`
\end{definition}

\begin{definition}[Configuration] A configuration of an automaton $A=\langle Ports,$ $Vars,Trans_G\rangle$ is defined as a tuple $(v_{loc},v_{adj})$ where $v_{loc}\in Val(Vars)$ is a valuation on local variables, and $v_{adj}\in Val(Adj(P))$ is a valuation on adjoint variables. We use $Conf(A)$ to denote all the configurations of $A$.
\end{definition}

\subsection{Canonical Form of Transitions and Automata}
\label{subsec:canonical}

\begin{definition}[Canonical Transitions]
A transition $t=g\rightarrow\{s_1,\cdots,s_n\}$ is canonical iff. its statements $\{s_i\}$ is an interleaving sequence of assignments and performs which starts from and ends by assignments, e.g. \texttt{a := exp$_1$; perform A; b := exp$_2$; $\cdots$ \texttt{c := exp$_3$}}.
\end{definition}

Assume $g\rightarrow\{s_1,\cdots,s_n\}$ is a transition of automaton $A$, the following algorithm shows how it is canonicalized,
\begin{description}
    \item[S1] Any continuous subsequences $s_i,\cdots,s_j(j>i)$ exists where all elements are assignment statements should be merged. As mentioned before, an assignment statement is represented as a function $f:Conf(A)\rightarrow Conf(A)$. Thus a list of multiple assignments $s_i,\cdots, s_j$ can be replaced using $s'=s_i\circ\cdots \circ s_j$.
    \item[S2] Keep going with \emph{S1} until there is no further subsequence to merge.
    \item[S3] Put identical assignment $id_{Conf(A)}$ into any two adjacent \emph{perform}s. Similarly, if the statements start from or end with a \emph{perform} statement, we should also use $id_{Conf(A)}$ to decorate its head and tail.
\end{description}

\begin{definition}[Canonical Automata]
    An automaton $A=\langle Ports,$ $Vars,Trans_G\rangle$ is canonical iff. a) $Trans_G$ includes only one transition group and b) all transitions in this group are also canonical.
\end{definition}
% \smalltitle{Reducing Order} First we consider a simpler situation, where only \emph{single} transitions are involved. Given a set of ordered transitions
% \[
%     \{g_1\rightarrow S_1,g_2\rightarrow S_2,\cdots,g_n\rightarrow S_n\}
% \]
% As required by the \emph{priority} assumption, a transition can be fired only if all the previous ones are not enabled (i.e. their guards are not satisfied) yet. In \lang{}, this feature is resolved simply by adding $\lnot g_i$ to all $g_j(j>i)$. E.g.
% \[
%     \{g_1\rightarrow S_1, g_2\land(\lnot g_1)\rightarrow S_2,\cdots,g_n\land(\lnot g_1\land \lnot g_2\land\cdots\land \lnot g_{n-1})\rightarrow S_n\}
% \]

Now we show how $Trans_G$ is reformed to make the automaton canonical.
Assume that $Trans_G$ can be represented by,
\begin{small}
\[
    \{t_{G_1}=\{g_{11}\rightarrow S_{11},\cdots, g_{1l_1}\rightarrow S_{1l_1}\},\cdots,t_{G_n}=\{g_{n1}\rightarrow S_{n1},\cdots,g_{nl_n}\rightarrow S_{nl_n}\}\}
\]
\end{small}

Informally speaking, once a transition in $t_{G_1}$ is enabled, all the other transitions in $t_{G_i}(i>1)$ should be strictly prohibited from being fired. We use $enab(t_G)$ to denote the condition where at least one transition in $t_G$ is enabled, formalized as
\[
    enab(t_G=\{g_1\rightarrow S_1,\cdots, g_n\rightarrow S_n\}) = g_1\lor\cdots\lor g_n
\]
Then we can generate the new set of transitions with no dependency on priority:
\begin{eqnarray*}
    g_{11}\rightarrow S_{11},&\cdots&,g_{1l_1}\rightarrow S_{1l_1}, \\
    g_{21}\land \lnot enab(t_{G_1})\rightarrow S_{21}, &\cdots&, g_{2l_2} \land \lnot enab(t_{G_1})\rightarrow S_{2l_2}, \cdots \\
    g_{n1}\land \lnot enab(t_{G_1},\cdots,t_{G_{n-1}})\rightarrow S_{n1}, &\cdots&, g_{nl_n} \land \lnot enab(t_{G_1},\cdots,t_{G_{n-1}})\rightarrow S_{nl_n} \\
\end{eqnarray*}
where $enab(t_{G_1},\cdots,t_{G_{n-1}})$ is an abbreviation of $enab(t_{G_1})\lor\cdots\lor enab(t_{G_{n-1}})$. It indicates that at least one group in $t_{G_i}$ is enabled.
\subsection{From System to Automaton}
\label{subsec:composition}

As mentioned previously, system in \lang{} provides an approach to combine automata through \emph{components} and \emph{connectors}. However, a system is not a semantics element in our framework, in other words, behavior of a system relies on how the automata are organized and scheduled. In this section, we present the algorithms that formally describe the composition approach in \lang{}.

Algorithm \ref{alg:compose}. shows how to construct the skeleton of target automaton. When flatting a \emph{system}, first we rename all the variables in this sub-automata (including both components and connectors) to avoid name conflicts. Next we simply copy all the internal transitions to the target automaton. 

\begin{algorithm}[H]
    \caption{Flatting Systems}
    \label{alg:compose}
    \begin{algorithmic}[1]
        \REQUIRE A system $S=\langle Ports, Automata, Internals, Links\rangle$
        \ENSURE An automaton $A$
        \STATE $A \leftarrow $ empty automaton
        \STATE $A.Ports\leftarrow S.Ports$
        \STATE rename \emph{local variables} in $Automata=\{A_1,\cdots,A_n\}$ to avoid duplicated names
        \STATE $ext\_trans\leftarrow \{\}$
        \FOR{$i\leftarrow 1,2,\cdots,n$}
            \STATE $A.Vars\leftarrow A.Vars + A_i.Vars$
            \STATE $A.Trans_G \leftarrow A.Trans_G + Internal(A_i.Trans_G)$
            \STATE $ext\_trans \leftarrow ext\_trans + External(A_i.Trans_G)$
        \ENDFOR
        \FOR{$set\_trans\in 2^{ext\_trans}$}
            \STATE $new\_edge\leftarrow \mbox{\texttt{Schedule}}(S, set\_trans)$ 
            \IF{\emph{new\_edge $\neq$ null}}
                \STATE $A.Trans_G = A.Trans_G + \{new\_edge\}$
            \ENDIF
        \ENDFOR
    \end{algorithmic}
\end{algorithm}

External transitions, on the other hand, have to synchronize with its corresponding external transitions in other automata. For example, when an automaton want to read some thing from a input port $P_1$, there must be another one that is writing something to its output port $P_2$ where $P_1$ and $P_2$ are overlapped in the system.

In \lang{} \emph{systems}, only adjoint variables (\texttt{reqRead, reqWrite and value}) are shared between automata. During synchronization, the most important principle is to make sure assignments to shared variables are executed before dereferencing them. The detailed algorithm is described in Algorithm \ref{alg:synchronize}.

\begin{algorithm}[t]
    \caption{Scheduling in a Synchronous Set of External Transitions}
    \label{alg:synchronize}
    \begin{algorithmic}[1]
        \REQUIRE A System $S$, a set of transitions $t_1,t_2,\cdots,t_n$ (in canonical form)
        \ENSURE A synchronized transition $t$
        \IF{$\{t_i\}$ don't belong to different automata or $\exists t_i$ is internal}
            \STATE $t\leftarrow null$
            \RETURN
        \ENDIF
        \STATE \emph{t.g, t.S} $\leftarrow \bigwedge_i t_i.g,\:\{\}$
        \STATE \emph{shared\_ports}$\leftarrow$ all the 
        \STATE 
        \STATE  \emph{G} $\leftarrow$ a Graph $\langle V,E\rangle$
                \COMMENT{create a dependency graph}
        \FOR{$i\leftarrow 1,\cdots,n$}
            \STATE add $\bot_i, \top_i$ to \emph{G.V}
            \STATE \emph{lasts} $\leftarrow\{\bot_i\}$
            \FOR {$j\leftarrow 1, 3, \cdots, len(t_i.S) - 1$}
                \STATE \emph{ports} $\leftarrow$ all the performed ports in $t_i.S_{j+1}$
                \FOR {$l\in lasts, p\in ports$}
                    \IF {$p\not\in G.V$}
                        \STATE add $p$ to $G.V$
                    \ENDIF
                    \STATE add edge $l \xrightarrow{t_i.S_j} p$ to \emph{G.E}
                \ENDFOR 
            \ENDFOR
            \FOR {$l\in lasts$}
                \STATE add edge $l \xrightarrow{t_i.S_{len(t_i.S)}} \top_i$ to \emph{G.E}
            \ENDFOR 
        \ENDFOR
        \STATE
        \IF{($G$ comprises a ring) \emph{or} ($\exists v\in G.v$ is a port whose \emph{degree} $\neq 4$)}
            \STATE $t\leftarrow null$
        \ELSE
            \STATE $t.S\leftarrow\{$ select all the statements in $G.E$ using topological sort $\}$
            \STATE replace \texttt{perform P} in $t.S$ with \texttt{P.reqRead, P.reqWrite := false, false}
        \ENDIF
    \end{algorithmic}
\end{algorithm}

Line 25 shows several situations where the synchronization process fails,
\begin{enumerate}
    \item The generated graph includes a \emph{ring}, which is a sign of \emph{circular dependencies}. For example, in one transition \texttt{perform A} shows up earlier than \texttt{perform B}, but in another transition the order is reversed.
    \item The generated graph includes a non-trivial vertex (labelled by a perform statement) whose degree is not equal to $4$. In other words, a port is not properly synchronized or synchronized with more than two transitions. 
    % TODO:
\end{enumerate}

Topological sorting, as we all knows, may generate different schedules for the same graph. The following theorem shows that all these schedules are equivalent as transition statements.

\begin{theorem}[Equivalence between Schedules] If two set of assignment statements $S_1, S_2$ are generated from the same set of external transitions, they have exactly the same behavior (i.e. execution of $S_1$ and $S_2$ under the same configuration will lead to the same result.
\end{theorem}

\subsection{Automaton as Labelled Transition System}

\begin{definition}[Transition System, TS]
    A transition system is a tuple $(S,\rightarrow)$ where $S$ is a set of states and $\rightarrow\subseteq S\times\Sigma\times S$ is a set of transitions. For simplicity reasons, we use $s\rightarrow s'$ to denote $(s,s')$ in $\rightarrow$.
\end{definition}

Suppose $A=\langle Ports, Vars, Trans_G\rangle$ is an automaton, its semantics can be captured by a labelled transition system $\langle S_A, \rightarrow_A\rangle$ where
\begin{itemize}
    \item $S_A$ is the set of all configurations of $A$.
    \item $\rightarrow_A\subseteq S_A\times \Sigma_A\times S_A$ is a set of transitions constructed by the following rules.
\end{itemize}

\begin{mathpar}
    \inferrule* [right=R-InputStatus] {p\in P_{in}}{(v_{loc}, v_{adj})\xrightarrow{}_A(v_{loc},v_{adj}[p.reqWrite\mapsto \lnot p.reqWrite])} \\
    \inferrule* [right=R-InputValue] {p\in P_{in}, val\in type(p.value)}{(v_{loc}, v_{adj})\xrightarrow{}_A(v_{loc},v_{adj}[p.value\mapsto val])} \\
    \inferrule* [right=R-OutputStatus] {p\in P_{out}}{(v_{loc}, v_{adj})\xrightarrow{}_A(v_{loc},v_{adj}[p.reqRead\mapsto \lnot p.reqRead])} \\
    \inferrule* [right=R-Internal] {\{g\rightarrow \{s\}\}\in Trans_G\mbox{ is internal}}{(v_{loc}, v_{adj})\xrightarrow{}_A s(v_{loc},v_{adj})} \\
    \inferrule* [right=R-External] {\{g\rightarrow S\}\in Trans_G\mbox{ is external, } \{s_1,\cdots,s_n\}\mbox{ are the assignments in $S$}}{(v_{loc}, v_{adj})\xrightarrow{}_A s_n\circ\cdots\circ s_1(v_{loc},v_{adj})} \\
\end{mathpar}

The first three rules describe the potential change of context, i.e. the adjoint variables. R-InputStatus and R-OutputStatus shows that the reading status of an output port and status of an input port may changed randomly. And R-InputValue shows that the value of an input port may be updated by the context.

The rule R-Internal models the internal transitions in $Trans_G$. As illustrated previously, an internal transition doesn't contains any perform statement. So its canonical form comprises only one assignment $s$. Firing such a transition will simply apply $s$ to the current configuration.

Meanwhile, R-External models the external transitions, where the automaton need to interact with its context. Fortunately, since all the context change are captured by the first three rules, we can simply regard the context as a set of local variables. Consequently, the only difference between an internal transition and an external transitions is that the later may contains multiple assignments.